\chapter{Αναγνώριση περισσότερων σημάτων}

Για την υποστήριξη πολλών σημάτων στο \textit{SignalsArray} χρησιμοποιούμε 1 από 
τις εξής 3 συναρτήσεις:
\begin{description}
	\item[ChangeDetector()]: τροποποιημένη έκδοση της αρχικής 
	\textit{ChangeDetector()} που ελέγχει 1 στοιχείο του πίνακα σύμφωνα με τα 
	ορίσματά της.
	\item[MultiChangeDetector()]: ελέγχει έναν αριθμό στοιχείων σύμφωνα με τα 
	ορίσματά της.
	\item[BitfieldChangeDetector()]: ελέγχει ένα αριθμό bitfields που το καθένα 
	αντιπροσωπεύει \textit{INT\_BIT} στοιχεία του πίνακα (όπου \textit{INT\_BIT} 
	ο 
	αριθμός bit σε έναν int).
\end{description}

\section{ChangeDetector()}

\begin{lstlisting}[language=c, caption={ChangeDetector()}, escapechar=$]
void* ChangeDetector(void* arg) {
    const parm* p = (parm*) arg;
    const int target = p->tid;$\label{line:ChangeDetector-target}$

    /* loop stops with pthread_cancel() call at main() */
    while (1) {
        /* use a temporary variable in order to load signalArray[target] once in
         * each loop */
        unsigned int t;
        /* active waiting until target value changes to 1 */
        while ((t = signalArray[target]) == oldValues[target]) {}

        oldValues[target] = t;
        if (t) {
            /* signal activated: 0->1 */
            struct timeval tv;
            gettimeofday(&tv, NULL);
            /*print current time in usecs since the Epoch. */
            printf("D %d %lu\n", target, (tv.tv_sec) * 1000000 + (tv.tv_usec));
        }
    }
}
\end{lstlisting}

Όπως φαίνεται στη \hyperref[line:ChangeDetector-target]{γραμμή 
\ref*{line:ChangeDetector-target}} η μεταβλητή \textit{target} παίρνει σταθερή 
τιμή κατά την εκτέλεση της συνάρτησης ίση με την τιμή του thread ID. Έτσι, 
ανοίγοντας αριθμό thread ίσο με τον αριθμό των στοιχείων του 
\textit{SignalsArray} μπορούμε να αναγνωρίζουμε όλες τις αλλαγές.

\section{MultiChangeDetector()}

\begin{lstlisting}[language=c, caption={MultiChangeDetector()}, escapechar=$]
void* MultiChangeDetector(void* arg) {
    const parm* p = (parm*) arg;
    const unsigned int tid = p->tid;
    const unsigned int start = tid * (N / NTHREADS)+$    
    \label{line:MultiChangeDetector-start}$
                               (tid < N % NTHREADS ? tid : N % NTHREADS);
    const unsigned int end = start + (N / NTHREADS) + (tid < N % NTHREADS);$
    \label{line:MultiChangeDetector-end}$
    unsigned int target = start;

    while (1) {
        unsigned int t;
        while ((t = signalArray[target]) == oldValues[target]) {
            target = (target < end - 1) ? target + 1 : start;
        }

        oldValues[target] = t;
        if (t) {
            struct timeval tv;
            gettimeofday(&tv, NULL);
            printf("D %u %lu\n", target, (tv.tv_sec) * 1000000 + (tv.tv_usec));
        }
    }
}
\end{lstlisting}

Με αυτή τη συνάρτηση το κάθε thread ελέγχει ένα σύνολο στοιχείων με το index 
\textit{target} να παίρνει τιμές από 
\hyperref[line:MultiChangeDetector-start]{\textit{start}} μέχρι 
\hyperref[line:MultiChangeDetector-end]{\textit{end}}

\section{BitfieldChangeDetector()}

\begin{lstlisting}[language=c, caption={BitfieldChangeDetector()}, escapechar=$]
void* BitfieldChangeDetector(void* arg) {
    parm* p = (parm*) arg;
    const unsigned int tid = p->tid;
    const unsigned int start = tid * (total_N / NTHREADS) +
                               (tid < total_N % NTHREADS ?
                                tid : total_N % NTHREADS);
    const unsigned int end = start + (total_N / NTHREADS) +
                             (tid < total_N % NTHREADS);
    unsigned int target = start;

    while (1) {
        unsigned int t;
        while ((t = signalArray[target]) == oldValues[target]) {
            target = (target < end - 1) ? target + 1 : start;
        }

        const int bit_idx = msb_changed(t, oldValues[target]);
        const unsigned int actual = bit_idx + INT_BIT * target;
        /* oldValues[target] = t; <-- this way we lose signal changes
         * when 2 or more signals change at the same time within a bitfield. */
        /* if multiple changes happen then msb_changed() each time will find
         * the change at the most significant bit
         * because ceil(log2(x)) is the MSB of x */
        oldValues[target] ^= 1 << bit_idx;

        if ((t >> bit_idx) & 1) {
            struct timeval tv;
            gettimeofday(&tv, NULL);
            printf("D %u %lu\n", actual, (tv.tv_sec) * 1000000 + (tv.tv_usec));
        }
    }
}
\end{lstlisting}

Το index \textit{target} παίρνει και πάλι με τον ίδιο τρόπο τιμές.
Η χρήση αυτής της συνάρτησης προϋποθέτει την αλλαγή του κώδικα της 
\textit{SignalReader()} 
καθώς διαιρείται το μέγεθος του \textit{SignalArray} κατά \textit{INT\_BIT}. 
Συγκεκριμένα, χρησιμοποιείται η μεταβλητή 
\hyperref[lst:total_N]{\textit{use\_bitfields}} για την αλλαγή του μεγέθους του 
πίνακα και η συνάρτηση \hyperref[lst:togle_signal]{\textit{toggle\_signal()}} 
για την 
εναλλαγή των σημάτων του πίνακα.
\begin{lstlisting}[language=c, caption={Αλλαγή μεγέθους του SignalArray}, 
escapechar=$, label={lst:total_N}]
if (use_bitfields) {
    target_function = BitfieldChangeDetector;
    open_threads = NTHREADS;
    total_N = N / INT_BIT + (N % INT_BIT != 0);
}
\end{lstlisting}
\begin{lstlisting}[language=c, caption={toggle\_signal()}, 
escapechar=$, label={lst:togle_signal}]
int toggle_signal(int r) {
    /* Toggles the value of signal r.
     * timeStamp[r] is updated before the signal actually changes it's value.
     * Otherwise, the detectors can detect the change with before timeStamp is 
     updated. */

    if (use_bitfields) {
        const int array_idx = r / INT_BIT;
        const int bit_idx = r % INT_BIT;
        const int return_value = !((signalArray[array_idx] >> bit_idx) & 1);

        gettimeofday(&timeStamp[r], NULL);
        signalArray[array_idx] ^= 1 << bit_idx;

        return return_value;
    } else {
        gettimeofday(&timeStamp[r], NULL);
        return signalArray[r] ^= 1;
    }
}
\end{lstlisting}