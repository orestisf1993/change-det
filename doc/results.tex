\chapter{Μετρήσεις}
Στο \hyperref[fig:1compare]{σχήμα \ref*{fig:1compare}} φαίνεται το ποσοστό των σημάτων τα οποία ανιχνεύτηκαν σε χρόνο μικρότερο ή ίσο του 1μsec.

\begin{figure}[h]
    \centering
    \subfloat[χωρίς χρήση acknoweldged]{{\includegraphics[width=0.5\textwidth]{plot/1compare_00_10.pdf}}}%
    \subfloat[με χρήση acknoweldged]{{\includegraphics[width=0.5\textwidth]{plot/1compare_10_10.pdf}}}%
    \caption{Ποσοστό αποκρίσεων που είναι μικρότερες του 1 μsec για την ανίχνευση N σημάτων με TIME\_MULTIPLIER = 10.}
    \label{fig:1compare}
\end{figure}

Στα γραφήματα του \hyperref[fig:delay]{σχήματος \ref*{fig:delay}}
φαίνεται η εξέλιξη των μέσων τιμών της απόκρισης ως προς το \lstinline!N!. 

\begin{figure}[h]
    \centering
    \subfloat[χωρίς χρήση acknoweldged]{{\includegraphics[width=0.5\textwidth]{plot/delay00_1.pdf}}}
    \subfloat[με χρήση acknoweldged]{{\includegraphics[width=0.5\textwidth]{plot/delay10_1.pdf}}}
    
    \subfloat[χωρίς χρήση acknoweldged]{{\includegraphics[width=0.5\textwidth]{plot/delay00_10.pdf}}}
    \subfloat[με χρήση acknoweldged]{{\includegraphics[width=0.5\textwidth]{plot/delay10_10.pdf}}}
    
    \subfloat[χωρίς χρήση acknoweldged]{{\includegraphics[width=0.5\textwidth]{plot/delay00_100.pdf}}}
    \subfloat[με χρήση acknoweldged]{{\includegraphics[width=0.5\textwidth]{plot/delay10_100.pdf}}}
    \caption{Μέση απόκριση (μsec) για την ανίχνευση N σημάτων σε ημι-λογαριθμική κλίμακα.}
    \label{fig:delay}
\end{figure}

Στις υπόλοιπες μετρήσεις η επιλογή της χρήσης συνάρτησης detector γίνεται αυτόματα.
Στο \hyperref[fig:bit_multi]{σχήμα \ref*{fig:bit_multi}} φαίνεται η διαφορά στις επιδόσεις των
\lstinline!MultiChangeDetector())! και \lstinline!BitfieldChangeDetector()! όταν χειροκίνητα επιλέγουμε μια από τις δύο.

\begin{figure}[h]
    \centering
    \includegraphics[width=\textwidth]{plot/delay_bit_mult.pdf}
    \caption{Σύγκριση των συναρτήσεων \lstinline!MultiChangeDetector())! και \lstinline!BitfieldChangeDetector()!}
    \label{fig:bit_multi}
\end{figure}