\chapter{Προβλήματα συγχρονισμού}

Στην αρχική έκδοση του κώδικα δεν μπορεί να γίνει αλλαγή σημάτων σε διάστημα 
μικρότερο των 0.1 δευτερολέπτων. Στις σύγχρονες CPU ο χρόνος αυτός είναι 
υπεραρκετός για να εντοπιστεί η αλλαγή του σήματος σύμφωνα με τις συναρτήσεις 
που 
αναφέρθηκαν. Ωστόσο, καθώς ο ελάχιστος χρόνος αναμονής μειώνεται, αυξάνεται η 
πιθανότητα να "χαθούν" μερικές αλλαγές σημάτων.

Αυτό συμβαίνει καθώς είναι δυνατό η \lstinline!SensorSignalReader()! να 
ενεργοποιήσει μια τιμή ενός σήματος και, πριν την ανιχνεύσει ένας detector, να 
την απενεργοποιήσει αμέσως μετά. Αυτό κυρίως συμβαίνει για μικρές τιμές του \lstinline!N!
καθώς για μεγαλύτερες οποιαδήποτε κλίση της \textit{usleep()} αρκεί για να 
ανιχνευθούν οι αλλαγές.

Παραλείποντας τελείως την κλήση της \lstinline!usleep()! παίρνουμε τις παρακάτω 
μετρήσεις για τα ποσοστά των λαθών (αριθμός ενεργοποιήσεων σημάτων προς αριθμό 
ανιχνεύσεων).

\begin{center}
\includegraphics[width=\textwidth]{plot/errorsp.pdf}
\label{fig:errorsp}
\end{center}

%\begin{figure}%
%    \centering
%    \subfloat{{\includegraphics[width=0.4\textwidth]{plot/errorsp.pdf}}}%
%    \qquad
%    \subfloat{{\includegraphics[width=0.4\textwidth]{plot/errorsp.pdf}}}%
%%    \caption{2 Figures side by side}%
%    \label{fig:example}%
%\end{figure}
\section{Εξάλειψη σφαλμάτων}

Τροποποιούμε τις συναρτήσεις των detectors έτσι ώστε να μηδενίσουμε τα ποσοστά των λαθών.
 Αυτό γίνεται με την χρήση ενός συστήματος επικοινωνίας των detectors με την 
\lstinline!SensorSignalReader()! και επομένως απαιτεί και την αλλαγή της τελευταίας.

Για ευκολία compile με και χωρίς την χρήση του συστήματος αυτού, τα σχετικά κομμάτια του κώδικα περιλαμβάνονται μέσα στο macro \hyperref[lst:use_ack]{\lstinline!USE_ACK()!} το οποίο διαγράφει αυτά τα τμήματα αν δεν είναι ορισμένο το flag \lstinline!USE_ACKNOWLEDGEMENT!.

\begin{lstlisting}[language=c, caption={Ορισμός του USE\_ACK()}, label={lst:use_ack}]
#ifdef USE_ACKNOWLEDGEMENT
    #define USE_ACK(x) x
#else
    #define USE_ACK(x)
#endif
\end{lstlisting}

Χρησιμοποιείται ένα επιπλέον array \lstinline!acknowledged! με \lstinline!N! στοιχεία για την επικοινωνία μεταξύ των threads το οποίο είναι αρχικά ενεργοποιημένο (όλα τα στοιχεία ίσα με 1).
Όπως φαίνεται στη 
\hyperref[line:detector-ack]{γραμμή \ref*{line:detector-ack}}
της
\hyperref[lst:ChangeDetector-ack]{καταχώρησης \ref*{lst:ChangeDetector-ack}}
όταν ανιχνεύεται μια αλλαγή σε ένα σήμα, το detector thread ενεργοποιεί το στοιχείο στη θέση \lstinline!target! του πίνακα \lstinline!acknowledged!.
Οι συναρτήσεις 
\lstinline!MultiChangeDetector()! και 
\lstinline!BitfieldChangeDetector()! δουλεύουν με ανάλογο τρόπο.

\begin{lstlisting}[language=c, caption={ChangeDetector() με χρήση array acknowledged}, escapechar=$, label={lst:ChangeDetector-ack}]
void* ChangeDetector(void* arg) {
    const parm* p = (parm*) arg;
    const unsigned int target = p->tid;

    /* loop stops with pthread_cancel() call at main() */
    while (1) {
        /* use a temporary variable in order to load signalArray[target] once in
         * each loop */
        unsigned int t;
        /* active waiting until target value changes to 1 */
        while ((t = signalArray[target]) == oldValues[target]) {}

        oldValues[target] = t;
        if (t) {
            /* signal activated: 0->1 */
            struct timeval tv;
            gettimeofday(&tv, NULL);
            /* print current time in usecs since the Epoch. */
            printf("D %d %lu\n", target, (tv.tv_sec) * MILLION + (tv.tv_usec));
        }

        USE_ACK(acknowledged[target] = 1;)$\label{line:detector-ack}$
    }
}
\end{lstlisting}

Όταν το \hyperref[lst:reader-ack]{\lstinline!SensorSignalReader()!}
επιλέγει ένα τυχαίο σήμα \lstinline!r!
\hyperref[line:reader-wait-ack]{ελέγχει} αν το αντίστοιχο στοιχείο στον πίνακα 
\lstinline!acknowledged! είναι ενεργοποιημένο.
Αν δεν είναι, περιμένει μέχρι την ενεργοποίησή του και τελικά το ξαναμηδενίζει.

\begin{lstlisting}[language=c, caption={SensorSignalReader() με χρήση array acknowledged}, escapechar=$, label={lst:reader-ack}]
void* SensorSignalReader(void* arg) {
    UNUSED(arg);
    srand(time(NULL));

    while (changing_signals) {
        // t in [1, 10]
        const unsigned int t = rand() % 10 + 1;
        if (time_multiplier) {
            usleep(t * time_multiplier);
        }

        const unsigned int r = rand() % N;

        USE_ACK(while (!acknowledged[r]) {})$\label{line:reader-wait-ack}$
        USE_ACK(acknowledged[r] = 0;)

        if (toggle_signal(r)) {
            printf("C %d %lu\n", r, (timeStamp[r].tv_sec) * MILLION +
                   (timeStamp[r].tv_usec));
        }
    }
    
    pthread_exit(NULL);
}
\end{lstlisting}

Με αυτόν τον τρόπο, δεν παράγονται λάθη για κανένα συνδυασμό ορισμάτων.
Ωστόσο, καθώς υπάρχει δυνατότητα καθυστέρησης της \lstinline!SensorSignalReader()! επηρεάζεται ο αριθμός των σημάτων που αλλάζουν.

\begin{center}
	\includegraphics[width=\textwidth]{plot/changes}
\end{center}