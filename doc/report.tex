%&preamble
% Save static part at preamble.tex and use command:
% xelatex -ini -shell-escape -job-name="preamble" "&xelatex preamble.tex\dump"
% to produce preamble.fmt

% Only for xelatex and lualatex. It provides an automatic and unified interface to feature-rich AAT and OpenType fonts.
% https://ctan.org/pkg/fontspec
\usepackage{fontspec}
\setmainfont{DejaVu Serif}
\renewcommand{\contentsname}{Περιεχόμενα}
\renewcommand{\listfigurename}{Λίστα Σχημάτων}
\renewcommand{\figurename}{Σχήμα}
\renewcommand{\lstlistingname}{Καταχώρηση}
\renewcommand{\lstlistlistingname}{List of \lstlistingname s}

\title{Εργασία στα Ενσωματωμένα Συστήματα Πραγματικού Χρόνου}
\author{Ορέστης Φλώρος-Μαλιβίτσης\\
  Τομέας Ηλεκτρονικής,\\
  Τμήμα Ηλ. Μηχανικών / Μηχανικών ΗΥ,\\
  Αριστοτέλειο Πανεπιστήμιο Θεσσαλονίκης}

\begin{document}
\maketitle
\tableofcontents
%\listoffigures
\newpage

% Δομή του Project
\chapter*{Δομή του Project} \label{project-structure}

\begin{description}
	\item[doc/] Φάκελος με τα *.tex αρχεία για την παραγωγή της αναφοράς.
	\item[doc/plot/] Φάκελος με τα αρχεία των γραφημάτων.
	\item[assigment] Η εκφώνηση.
	\item[pace.c] Ο κώδικας του προγράμματος σε c.
	\item[presentation.pdf] Η παρουσίαση που έγινε στο μάθημα σχετικά με το seL4.
	\item[report.pdf] Αυτή η αναφορά.
	\item[script.py] Script σε python3 για την παραγωγή δεδομένων και τον έλεγχό τους.
\end{description}

\chapter{Εισαγωγή}

Στην άσκηση αυτή αναζητούμε την οριακή συμπεριφορά του υπολογιστή σε πραγματικό 
χρόνο.

Θεωρούμε ένα array \textit{SignalsArray} από signals που έχουν τιμή 0 
ή 1 ενώ κατά την εκκίνηση του προγράμματος είναι όλα μηδενισμένα. Στην αρχική 
έκδοση του 
κώδικα ένα thread, που τρέχει την συνάρτηση \textit{SensorSignalReader()}, 
αλλάζει τυχαία στοιχεία του array 
\textit{SignalsArray} και ένα άλλο thread τρέχει την συνάρτηση 
\textit{ChangeDetector()} που προσπαθεί να ανιχνεύσει τις αλλαγές στο array.

Παρουσιάζεται η αρχική έκδοση του κώδικα:

\lstinputlisting[language=c, caption=Αρχική έκδοση του pace.c]{initial_pace.c}

Ζητείται η επέκταση του κώδικα ώστε να είναι πιθανή η αναγνώριση περισσότερων 
σημάτων.
\chapter{Αναγνώριση περισσότερων σημάτων}

Για την υποστήριξη πολλών σημάτων στο \lstinline!SignalsArray! χρησιμοποιούμε 1 από 
τις εξής 3 συναρτήσεις:
\begin{description}
	\item[ChangeDetector()]: τροποποιημένη έκδοση της αρχικής 
	\lstinline!ChangeDetector()! που ελέγχει 1 στοιχείο του πίνακα σύμφωνα με τα 
	ορίσματά της.
	\item[MultiChangeDetector()]: ελέγχει έναν αριθμό στοιχείων σύμφωνα με τα 
	ορίσματά της.
	\item[BitfieldChangeDetector()]: ελέγχει ένα αριθμό bitfields που το καθένα 
	αντιπροσωπεύει \lstinline!INT\_BIT! στοιχεία του πίνακα (όπου \lstinline!INT\_BIT! 
	ο 
	αριθμός bit σε έναν int).
\end{description}

\section{ChangeDetector()}

\begin{lstlisting}[language=c, caption={ChangeDetector()}, escapechar=$]
void* ChangeDetector(void* arg) {
    const parm* p = (parm*) arg;
    const unsigned int target = p->tid;$\label{line:ChangeDetector-target}$

    /* loop stops with pthread_cancel() call at main() */
    while (1) {
        /* use a temporary variable in order to load signalArray[target] once in
         * each loop */
        unsigned int t;
        /* active waiting until target value changes to 1 */
        while ((t = signalArray[target]) == oldValues[target]) {}

        oldValues[target] = t;
        if (t) {
            /* signal activated: 0->1 */
            struct timeval tv;
            gettimeofday(&tv, NULL);
            /* print current time in usecs since the Epoch. */
            printf("D %d %lu\n", target, (tv.tv_sec) * MILLION + (tv.tv_usec));
        }
    }
}
\end{lstlisting}

Όπως φαίνεται στη \hyperref[line:ChangeDetector-target]{γραμμή 
\ref*{line:ChangeDetector-target}} η μεταβλητή \textit{target} παίρνει σταθερή 
τιμή κατά την εκτέλεση της συνάρτησης ίση με την τιμή του thread ID. Έτσι, 
ανοίγοντας αριθμό thread ίσο με τον αριθμό των στοιχείων του 
\textit{SignalsArray} μπορούμε να αναγνωρίζουμε όλες τις αλλαγές. Χρησιμοποιείται 
το array \textit{oldValues} για να αποθηκεύονται οι παλιές τιμές των σημάτων.

\section{MultiChangeDetector()}

\begin{lstlisting}[language=c, caption={MultiChangeDetector()}, escapechar=$]
void* MultiChangeDetector(void* arg) {
    const parm* p = (parm*) arg;
    const unsigned int tid = p->tid;
    const unsigned int start = tid * (N / requested_threads) +
                               (tid < N % requested_threads ? tid : N %$ 
                               $ requested_threads);$    
    \label{line:MultiChangeDetector-start}$
    const unsigned int end = start + (N / requested_threads) + (tid < N % requested_threads);$
    \label{line:MultiChangeDetector-end}$
    unsigned int target = start;

    while (1) {
        unsigned int t;
        while ((t = signalArray[target]) == oldValues[target]) {
            target = (target < end - 1) ? target + 1 : start;
        }

        oldValues[target] = t;
        if (t) {
            struct timeval tv;
            gettimeofday(&tv, NULL);
            printf("D %u %lu\n", target, (tv.tv_sec) * MILLION + (tv.tv_usec));
        }
    }
}
\end{lstlisting}

Με αυτή τη συνάρτηση το κάθε thread ελέγχει ένα σύνολο στοιχείων με το index 
\textit{target} να παίρνει τιμές από 
\hyperref[line:MultiChangeDetector-start]{\textit{start}} μέχρι 
\hyperref[line:MultiChangeDetector-end]{\textit{end}}

\section{BitfieldChangeDetector()}

\begin{lstlisting}[language=c, caption={BitfieldChangeDetector()}, escapechar=$]
void* BitfieldChangeDetector(void* arg) {
    parm* p = (parm*) arg;
    const unsigned int tid = p->tid;
    const unsigned int start = tid * (total_N / requested_threads) +
                               (tid < total_N % requested_threads ?
                                tid : total_N % requested_threads);
    const unsigned int end = start + (total_N / requested_threads) +
                             (tid < total_N % requested_threads);
    unsigned int target = start;

    while (1) {
        unsigned int t;
        while ((t = signalArray[target]) == oldValues[target]) {
            target = (target < end - 1) ? target + 1 : start;
        }

        const unsigned int bit_idx = msb_changed(t, oldValues[target]);$
        \label{line:BitfieldChangeDetector-msb}$
        const unsigned int actual = bit_idx + INT_BIT * target;
        /* oldValues[target] = t; <-- this way we lose signal changes
         * when 2 or more signals change at the same time within a bitfield. */
        /* if multiple changes happen then msb_changed() each time will find
         * the change at the most significant bit
         * because ceil(log2(x)) is the MSB of x */
        oldValues[target] ^= 1 << bit_idx;

        if ((t >> bit_idx) & 1) {
            struct timeval tv;
            gettimeofday(&tv, NULL);
            printf("D %u %lu\n", actual, (tv.tv_sec) * MILLION + (tv.tv_usec));
        }
    }
}
\end{lstlisting}

Το index \textit{target} παίρνει και πάλι με τον ίδιο τρόπο τιμές.
Η χρήση αυτής της συνάρτησης προϋποθέτει την αλλαγή του κώδικα της 
\textit{SignalReader()} 
καθώς διαιρείται το μέγεθος του \textit{SignalArray} κατά \textit{INT\_BIT}. 
Συγκεκριμένα, χρησιμοποιείται η μεταβλητή 
\hyperref[lst:total_N]{\textit{use\_bitfields}} για την αλλαγή του μεγέθους του 
πίνακα και η συνάρτηση \hyperref[lst:togle_signal]{\textit{toggle\_signal()}} 
για την εναλλαγή των σημάτων του πίνακα.
\begin{lstlisting}[language=c, caption={Αλλαγή μεγέθους του SignalArray}, 
escapechar=$, label={lst:total_N}]
if (use_bitfields) {
    target_function = BitfieldChangeDetector;
    open_threads = NTHREADS;
    total_N = N / INT_BIT + (N % INT_BIT != 0);
}
\end{lstlisting}
\begin{lstlisting}[language=c, caption={toggle\_signal()}, 
escapechar=$, label={lst:togle_signal}]
int toggle_signal(int r) {
    /* Toggles the value of signal r.
     * timeStamp[r] is updated before the signal actually changes it's value.
     * Otherwise, the detectors can detect the change before timeStamp is 
     updated. */

    if (use_bitfields) {
        const int array_idx = r / INT_BIT;
        const int bit_idx = r % INT_BIT;
        const int return_value = !((signalArray[array_idx] >> bit_idx) & 1);

        gettimeofday(&timeStamp[r], NULL);
        signalArray[array_idx] ^= 1 << bit_idx;

        return return_value;
    } else {
        gettimeofday(&timeStamp[r], NULL);
        return signalArray[r] ^= 1;
    }
}
\end{lstlisting}
Για την ανίχνευση των ενεργοποιήσεων χρησιμοποιείται η συνάρτηση 
\hyperref[lst:msb_changed]{\lstinline!msb\_changed()!}. H \lstinline!msb\_changed()!
χρησιμοποιεί 
\href{https://graphics.stanford.edu/~seander/bithacks.html#IntegerLogLookup}{lookup
 table} για τον γρηγορότερο υπολογισμό του log2. H \lstinline!msb\_changed()!
 καλείται στη
 \hyperref[line:BitfieldChangeDetector-msb]{γραμμή 
 \ref*{line:BitfieldChangeDetector-msb}} για την εύρεση της αλλαγής μεταξύ της 
 προηγούμενης και της τρέχουσας τιμής του bitfield.
 Τα ενεργοποιημένα bits του αποτελέσματος $A \oplus B$ είναι τα διαφορετικά bit 
 μεταξύ των bitfields $A$ και $B$.
 Έτσι, παίρνοντας κάθε φορά το ακέραιο μέρος του $\log_2 \left( A \oplus B 
 \right)$ 
 μπορούμε να βρίσκουμε το πιο σημαντικό αλλαγμένο bit μεταξύ 2 καταστάσεων.
\begin{lstlisting}[language=c, caption={msb\_changed()}, 
escapechar=$, label={lst:msb_changed}]
#define LT(n) n, n, n, n, n, n, n, n, n, n, n, n, n, n, n, n
static const char LogTable256[256] = {
    -1, 0, 1, 1, 2, 2, 2, 2, 3, 3, 3, 3, 3, 3, 3, 3,
    LT(4), LT(5), LT(5), LT(6), LT(6), LT(6), LT(6),
    LT(7), LT(7), LT(7), LT(7), LT(7), LT(7), LT(7), LT(7)
};

unsigned int msb_changed(unsigned int current, unsigned int old) {
    /* Use bit-wise XOR to find the different bits between signalArray[target] and
     * oldValues[target]. Return the most significant of them using log2.
     * Kinda faster than gcc's __builtin_clz() */
    /* diff is INT_BIT-bit word to find the log2 of */
    unsigned int diff = current ^ old;
    unsigned int t, tt;  /* temporaries */

    if ((tt = diff >> 16)) {
        return (t = tt >> 8) ? 24 + LogTable256[t] : 16 + LogTable256[tt];
    } else {
        return (t = diff >> 8) ? 8 + LogTable256[t] : LogTable256[diff];
    }
}
\end{lstlisting}
\chapter{Προβλήματα συγχρονισμού}

Στην αρχική έκδοση του κώδικα δεν μπορεί να γίνει αλλαγή σημάτων σε διάστημα 
μικρότερο των 0.1 δευτερολέπτων. Στις σύγχρονες CPU ο χρόνος αυτός είναι 
υπεραρκετός για να εντοπιστεί η αλλαγή του σήματος σύμφωνα με τις συναρτήσεις 
που 
αναφέρθηκαν. Ωστόσο, καθώς ο ελάχιστος χρόνος αναμονής μειώνεται, αυξάνεται η 
πιθανότητα να "χαθούν" μερικές αλλαγές σημάτων.

Αυτό συμβαίνει καθώς είναι δυνατό η \lstinline!SensorSignalReader()! να 
ενεργοποιήσει μια τιμή ενός σήματος και, πριν την ανιχνεύσει ένας detector, να 
την απενεργοποιήσει αμέσως μετά. Αυτό κυρίως συμβαίνει για μικρές τιμές του \lstinline!N!
καθώς για μεγαλύτερες οποιαδήποτε κλίση της \textit{usleep()} αρκεί για να 
ανιχνευθούν οι αλλαγές.

Παραλείποντας τελείως την κλήση της \lstinline!usleep()! παίρνουμε τις παρακάτω 
μετρήσεις για τα ποσοστά των λαθών (αριθμός ενεργοποιήσεων σημάτων προς αριθμό 
ανιχνεύσεων).

\begin{center}
\includegraphics[width=\textwidth]{plot/errorsp.pdf}
\label{fig:errorsp}
\end{center}

%\begin{figure}%
%    \centering
%    \subfloat{{\includegraphics[width=0.4\textwidth]{plot/errorsp.pdf}}}%
%    \qquad
%    \subfloat{{\includegraphics[width=0.4\textwidth]{plot/errorsp.pdf}}}%
%%    \caption{2 Figures side by side}%
%    \label{fig:example}%
%\end{figure}
\section{Εξάλειψη σφαλμάτων}

Τροποποιούμε τις συναρτήσεις των detectors έτσι ώστε να μηδενίσουμε τα ποσοστά των λαθών.
 Αυτό γίνεται με την χρήση ενός συστήματος επικοινωνίας των detectors με την 
\lstinline!SensorSignalReader()! και επομένως απαιτεί και την αλλαγή της τελευταίας.

Για ευκολία compile με και χωρίς την χρήση του συστήματος αυτού, τα σχετικά κομμάτια του κώδικα περιλαμβάνονται μέσα στο macro \hyperref[lst:use_ack]{\lstinline!USE_ACK()!} το οποίο διαγράφει αυτά τα τμήματα αν δεν είναι ορισμένο το flag \lstinline!USE_ACKNOWLEDGEMENT!.

\begin{lstlisting}[language=c, caption={Ορισμός του USE\_ACK()}, label={lst:use_ack}]
#ifdef USE_ACKNOWLEDGEMENT
    #define USE_ACK(x) x
#else
    #define USE_ACK(x)
#endif
\end{lstlisting}

Χρησιμοποιείται ένα επιπλέον array \lstinline!acknowledged! με \lstinline!N! στοιχεία για την επικοινωνία μεταξύ των threads το οποίο είναι αρχικά ενεργοποιημένο (όλα τα στοιχεία ίσα με 1).
Όπως φαίνεται στη 
\hyperref[line:detector-ack]{γραμμή \ref*{line:detector-ack}}
της
\hyperref[lst:ChangeDetector-ack]{καταχώρησης \ref*{lst:ChangeDetector-ack}}
όταν ανιχνεύεται μια αλλαγή σε ένα σήμα, το detector thread ενεργοποιεί το στοιχείο στη θέση \lstinline!target! του πίνακα \lstinline!acknowledged!.
Οι συναρτήσεις 
\lstinline!MultiChangeDetector()! και 
\lstinline!BitfieldChangeDetector()! δουλεύουν με ανάλογο τρόπο.

\begin{lstlisting}[language=c, caption={ChangeDetector() με χρήση array acknowledged}, escapechar=$, label={lst:ChangeDetector-ack}]
void* ChangeDetector(void* arg) {
    const parm* p = (parm*) arg;
    const unsigned int target = p->tid;

    /* loop stops with pthread_cancel() call at main() */
    while (1) {
        /* use a temporary variable in order to load signalArray[target] once in
         * each loop */
        unsigned int t;
        /* active waiting until target value changes to 1 */
        while ((t = signalArray[target]) == oldValues[target]) {}

        oldValues[target] = t;
        if (t) {
            /* signal activated: 0->1 */
            struct timeval tv;
            gettimeofday(&tv, NULL);
            /* print current time in usecs since the Epoch. */
            printf("D %d %lu\n", target, (tv.tv_sec) * MILLION + (tv.tv_usec));
        }

        USE_ACK(acknowledged[target] = 1;)$\label{line:detector-ack}$
    }
}
\end{lstlisting}

Όταν το \hyperref[lst:reader-ack]{\lstinline!SensorSignalReader()!}
επιλέγει ένα τυχαίο σήμα \lstinline!r!
\hyperref[line:reader-wait-ack]{ελέγχει} αν το αντίστοιχο στοιχείο στον πίνακα 
\lstinline!acknowledged! είναι ενεργοποιημένο.
Αν δεν είναι, περιμένει μέχρι την ενεργοποίησή του και τελικά το ξαναμηδενίζει.

\begin{lstlisting}[language=c, caption={SensorSignalReader() με χρήση array acknowledged}, escapechar=$, label={lst:reader-ack}]
void* SensorSignalReader(void* arg) {
    UNUSED(arg);
    srand(time(NULL));

    while (changing_signals) {
        // t in [1, 10]
        const unsigned int t = rand() % 10 + 1;
        if (time_multiplier) {
            usleep(t * time_multiplier);
        }

        const unsigned int r = rand() % N;

        USE_ACK(while (!acknowledged[r]) {})$\label{line:reader-wait-ack}$
        USE_ACK(acknowledged[r] = 0;)

        if (toggle_signal(r)) {
            printf("C %d %lu\n", r, (timeStamp[r].tv_sec) * MILLION +
                   (timeStamp[r].tv_usec));
        }
    }
    
    pthread_exit(NULL);
}
\end{lstlisting}

Με αυτόν τον τρόπο, δεν παράγονται λάθη για κανένα συνδυασμό ορισμάτων.
Ωστόσο, καθώς υπάρχει δυνατότητα καθυστέρησης της \lstinline!SensorSignalReader()! επηρεάζεται ο αριθμός των σημάτων που αλλάζουν.

\begin{center}
	\includegraphics[width=\textwidth]{plot/changes}
\end{center}
\chapter{Μετρήσεις}
Στο \hyperref[fig:1compare]{σχήμα \ref*{fig:1compare}} φαίνεται το ποσοστό των σημάτων τα οποία ανιχνεύτηκαν σε χρόνο μικρότερο ή ίσο του 1μsec.

\begin{figure}[h]
    \centering
    \subfloat[χωρίς χρήση acknoweldged]{{\includegraphics[width=0.5\textwidth]{plot/1compare_00_10.pdf}}}%
    \subfloat[με χρήση acknoweldged]{{\includegraphics[width=0.5\textwidth]{plot/1compare_10_10.pdf}}}%
    \caption{Ποσοστό αποκρίσεων που είναι μικρότερες του 1 μsec για την ανίχνευση N σημάτων με TIME\_MULTIPLIER = 10.}
    \label{fig:1compare}
\end{figure}

Στα γραφήματα του \hyperref[fig:delay]{σχήματος \ref*{fig:delay}}
φαίνεται η εξέλιξη των μέσων τιμών της απόκρισης ως προς το \lstinline!N!. 

\begin{figure}[h]
    \centering
    \subfloat[χωρίς χρήση acknoweldged]{{\includegraphics[width=0.5\textwidth]{plot/delay00_1.pdf}}}
    \subfloat[με χρήση acknoweldged]{{\includegraphics[width=0.5\textwidth]{plot/delay10_1.pdf}}}
    
    \subfloat[χωρίς χρήση acknoweldged]{{\includegraphics[width=0.5\textwidth]{plot/delay00_10.pdf}}}
    \subfloat[με χρήση acknoweldged]{{\includegraphics[width=0.5\textwidth]{plot/delay10_10.pdf}}}
    
    \subfloat[χωρίς χρήση acknoweldged]{{\includegraphics[width=0.5\textwidth]{plot/delay00_100.pdf}}}
    \subfloat[με χρήση acknoweldged]{{\includegraphics[width=0.5\textwidth]{plot/delay10_100.pdf}}}
    \caption{Μέση απόκριση (μsec) για την ανίχνευση N σημάτων σε ημι-λογαριθμική κλίμακα.}
    \label{fig:delay}
\end{figure}

Στις υπόλοιπες μετρήσεις η επιλογή της χρήσης συνάρτησης detector γίνεται αυτόματα.
Στο \hyperref[fig:bit_multi]{σχήμα \ref*{fig:bit_multi}} φαίνεται η διαφορά στις επιδόσεις των
\lstinline!MultiChangeDetector())! και \lstinline!BitfieldChangeDetector()! όταν χειροκίνητα επιλέγουμε μια από τις δύο.

\begin{figure}[h]
    \centering
    \includegraphics[width=\textwidth]{plot/delay_bit_mult.pdf}
    \caption{Σύγκριση των συναρτήσεων \lstinline!MultiChangeDetector())! και \lstinline!BitfieldChangeDetector()!}
    \label{fig:bit_multi}
\end{figure}

\end{document}
